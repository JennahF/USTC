\documentclass[UTF8]{ctexart}
\usepackage{graphicx}
\graphicspath{{img/}} 
\title{Homework\ 7}
\author{PB17111623}
\author{PB17111623 范睿}
\date{\today}
\usepackage[a4paper,bottom=3.5cm]{geometry}
\usepackage{algorithm}  
\usepackage{algorithmicx}
\usepackage{amsmath}  
\usepackage{algpseudocode}  %算法的包
\usepackage{amssymb}
\begin{document}
\maketitle
\section{}
\subsection{给定一个有向无环图G=(V,E),边权重为实数,给定图中两个顶点s和t。设计动态规划算法,求从s到t的最长加权简单路径。}
设weight[s][t]为s与t间的最长加权路径的权重,有递归式:\\
\begin{center}
$weight[s][t]=\max\limits_{p\in s.S}(weight[p][t]+w[s][t])$
\end{center}
其中,s.S表示所有s指向的结点的集合,w[s][t]表示st这条边的边权。

\begin{algorithm}
	\caption{Weight}
	\begin{algorithmic}[1]%显示行号
	\Require 图G,顶点s,顶点t (weight数组初始均为-1)
	\Ensure 最长边权w
	\If{s is t}
		\State{$weight[s][t]\gets 0$}\\
		\Return 0
	\EndIf
	\For{$u \in V$}
		\For{$v \in V$}
			\For{$p\ \in\ V\ and\ pq\ \in E$}
				\If{$weight[u][p]\ is\ -1$}
					\State{$weight[u][p]\gets Weight(G,u,p)$}
				\EndIf
				\State{$weight[u][v]=\max(path[u][j], w[p][j]+weight[i][p])$}
			\EndFor
		\EndFor
	\EndFor\\
	\Return weight[s][t]
	\end{algorithmic}
\end{algorithm}


\subsection{设定动态规划算法求解 0-1 背包问题,要求运行时间为 O(nW), n 为商品数量,W 是小偷能放进背
包的最大商品总重量。}

\begin{algorithm}
	\caption{}
	\begin{algorithmic}[1]%显示行号
	\Require 商品数量n,最大重量W,每个商品的重量和价值$w_i,v_i,1\le i\le n$
	\Ensure 最大价值
	\State{let A[0...W][1...n+1] be a new array}
	\For{i = 0 to W}
		\State{$A[i][n+1] \gets 0$}
	\EndFor
	\For{i = n down to 1}
		\For{j = 0 to W}
			\State{$A[i][j] \gets 0$}
			\If{$j < w_i$}
				\State{$A[i][j] \gets A[i][j+1]$}
			\Else
				\State{$value \gets A[i-w_i][j+1]+v_j$}
				\If{$value \ge A[i][j]$}
					\State{$A[i][j] \gets value$}
				\EndIf
			\EndIf
		\EndFor
	\EndFor
	\Return A[W][1]
	\end{algorithmic}
\end{algorithm}

\subsection{一位公司主席正在向 Stewart 教授咨询公司聚会方案。公司的内部结构关系是层次化的,即员工按主管-下属关系构成一棵树,根结点为公司主席。人事部按“宴会交际能力”为每个员工打分,分值为实数。为了使所有参加聚会的员工都感到愉快,主席不希望员工及其直接主管同时出席。公司主席向 Stewart 教授提供公司结构树,采用左孩子右兄弟表示法(参见课本 10.4 节)描述。每个节点除了保存指针外,还保存员工的名字和宴会交际评分。设计算法,求宴会交际评分之和最大的宾客名单。分析算法复杂度。}
寻找子结构:\\
令score(T)表示以T为根的树的评分最大值,有递归式:\\
\begin{center}
$score(T) = max(T.score+\sum_{p\in T's grandson}score(p), \sum_{p\in T's children}score(p))$
\end{center}

\begin{algorithm}
	\caption{}
	\begin{algorithmic}[1]%显示行号
	\Require 树T
	\Ensure 最大评分scoremax
	\State{将树T转换成堆的形式存在数组A[1...n]中,n为总人数}
	\State{let B[1...n] be a new array}
	\For{i = 1 to n}
		\State{$B[i] \gets 0$}
	\EndFor
	\For{i = n downto 1}
		\If{$lchild(i) > n$}
			\State{$B[i] \gets score_i$}
		\Else
			\If{$grandson(i) > n$}
				\State{$value \gets score_i+score_{rchild(i)}+score_{rchild(rchild(i))}...$}
				\State{$B[i] \gets value>score_{lchild(i)}?value:score_{lchild(i)}$}
			\Else
				\State{$value1 \gets score_i+score_{lchild(lchild(i))}$}
				\State{$value2 \gets score_{lchild(i)}$}
				\State{$B[i] \gets value1>value2?value1:value2$}
			\EndIf
		\EndIf
	\EndFor
	\Return B[1]
	\end{algorithmic}
\end{algorithm}
复杂度为$\mathcal{O}(n)$n为人数
\subsection{. 设计一个高效的算法,对实数线上给定的一个点集 {x1, x2,...xn}, 求一个单位长度闭区间的集合,包含所有给定的点,并要求此集合最小。证明你的算法是正确的。}
先找到最左边的点,将单位区间的左端点与之重合,用单位长度区间覆盖掉,然后把所有被覆盖掉的点去掉,再找最左边的点,重复,直到所有点全部被覆盖掉。
证明是最小:\\
1.最左边的点一定会被某一个区间覆盖\\
2. 如果从最左边的点的左边开始放置区间,则会将区间左边点至最左边的区域浪费掉,有更优解,即提出算法。\\

\subsection{考虑用最少的硬币找 n 美分零钱的问题。假定每种硬币的面额都是整数。设计贪心算法求解找零问题,假定有 25 美分、10 美分、5 美分和 1 美分四种面额的硬币。证明你的算法能找到最优解。}
先找最多的25美分,再找最多的10美分,再找最多的5美分,剩下的全用1美分找。\\
1. 同样价格的前,大面值美分的个数比小面值美分的个数少。
2. 如果将本算法中的大面值用小面值替换,将会得到更多的硬币,不是最优解。


\end{document}
