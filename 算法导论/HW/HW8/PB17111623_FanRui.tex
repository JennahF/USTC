\documentclass[UTF8]{ctexart}
\usepackage{graphicx}
\graphicspath{{img/}} 
\title{Homework\ 8}
\author{PB17111623}
\author{PB17111623 范睿}
\date{\today}
\usepackage[a4paper,bottom=3.5cm]{geometry}
\usepackage{algorithm}  
\usepackage{algorithmicx}
\usepackage{amsmath}  
\usepackage{algpseudocode}  %算法的包
\usepackage{amssymb}
\begin{document}
\maketitle
\subsection{假定我们对一个数据结构执行一个由 n 个操作组成的操作序列,当 i 严格为 2 的幂时,第 i 个操作的代价为 i,否则代价为 1. 使用聚合分析确定每个操作的摊还代价。}
\begin{center}
$totalcost=\sum\limits_{i=1}^{n}cost(i)\leq \sum\limits_{i=1}^{n}1+\sum\limits_{i=0}^{\lfloor\log_2 n\rfloor}(2^i-1)=n+(2n-1)-(\lfloor\log_2 n\rfloor+1)=3n-\lfloor\log_2 n\rfloor-2$\\
$averagecost=\mathcal{O}(\frac{3n-\lfloor\log_2 n\rfloor-2}{n})=\mathcal{O}(1)$
\end{center}
\subsection{用核算法重做第一题。}
重新定义每次操作的代价:\\
当 i 严格为 2 的幂时,第 i 个操作的代价为 2*i,其余代价为0。\\
新定义的代价下,n次操作总代价一定比题目中总代价大。
在新定义的代价下:
\begin{center}
$newtotalcost=\sum\limits_{i=0}^{\log_2 n}(2\times 2^i)=4n-2 \geq 3n-\lfloor\log_2 n\rfloor-2=totalcost $
\end{center}
在新定义的代价下,平均代价为:\\
\begin{center}
$averagecost=\mathcal{O}(\frac{4n-2}{n})=\mathcal{O}(1)$
\end{center}
\subsection{使用势能法重做第一题。}
定义$D(0)=0,\ D(i)=2i-2^{\lfloor\log_2 i\rfloor}$\\
$\forall n >0, D(n)>0$\\
在D(i)如此定义下,$c1'=c1+D(1)-D(0)=1$\\
$\forall i$不是2的整数次幂,$ci'=ci+D(i)-D(i-1)=3$\\
$\forall i$是2的整数次幂,\\$ci'=ci+D(i)-D(i-1)=i+2i-2^{1+\lfloor\log_2 i\rfloor}-(2(i-1)-2^{1+\lfloor\log_2 i\rfloor-1})=i+2-2^{1+\lfloor\log_2 i\rfloor-1}=3$\\
因此
\begin{center}
$newtotalcost=1+\sum\limits_{i=2}^{n}3=3n-2$\\
$averagecost=\mathcal{O}(1)$
\end{center}
\subsection{}
\subsubsection{a}
首先证明,利用沿着第1维计算n/n1个独立的一维DFT的结果可以计算出前两维的DFT:\\
沿着第一维计算的DFT的结果为
\begin{center}
$y_{k_1,j_2,...,j_d}=\sum\limits_{j_1=0}^{n_1-1}a_{j_1,j_2,...,j_d}\omega^{j_1k_1}_{n_1}$
\end{center}
对$y_{k_1,j_2,...,j_d}$在第二维上进行DFT,结果为
\begin{center}
$y_{k_1,k_2,...,j_d}=\sum\limits_{j_2=0}^{n_2-1}y_{k_1,j_2,...,j_d}\omega^{j_2k_2}_{n_2}=\sum\limits_{j_2=0}^{n_2-1}(\sum\limits_{j_1=0}^{n_1-1}a_{j_1,j_2,...,j_d}\omega^{j_1k_1}_{n_1})\omega^{j_2k_2}_{n_2}=\sum\limits_{j_2=0}^{n_2-1}\sum\limits_{j_1=0}^{n_1-1}a_{j_1,j_2,...,j_d}\omega^{j_1k_1}_{n_1}\omega^{j_2k_2}_{n_2}$
\end{center}
因此,有第1维的DFT结果可得前两维的DFT结果。\\
下面证明,由前i为的DFT结果可得前i+1维的DFT结果:\\
前i维的DFT结果为:\\
\begin{center}
$y_{k_1,k_2,...,k_i,j_{i+1},...,j_d}=\sum\limits_{j_1=0}^{n_1-1}\sum\limits_{j_2=0}^{n_2-1}...\sum\limits_{j_i=0}^{n_i-1}a_{j_1,j_2,...j_d}\omega^{j_1k_1}_{n_1}\omega^{j_2k_2}_{n_2}...\omega^{j_ik_i}_{n_i}$
\end{center}
对$y_{k_1,k_2,...,k_i,j_{i+1},...,j_d}$进行第i+1维的DFT,得到:\\
\begin{center}
$y_{k_1,k_2,...,k_i,k_{i+1},j_{i+2}...,j_d}=\sum\limits_{j_{i+1}=0}^{n_{i+1}-1}y_{k_1,k_2,...,k_i,j_{i+1},...,j_d}\omega^{j_{i+1}k_{i+1}}_{n_{i+1}}=\sum\limits_{j_{i+1}=0}^{n_{i+1}-1}(\sum\limits_{j_1=0}^{n_1-1}\sum\limits_{j_2=0}^{n_2-1}...\sum\limits_{j_i=0}^{n_i-1}a_{j_1,j_2,...j_d}\omega^{j_1k_1}_{n_1}\omega^{j_2k_2}_{n_2}...\omega^{j_ik_i}_{n_i})\omega^{j_{i+1}k_{i+1}}_{n_{i+1}}=\sum\limits_{j_{1}=0}^{n_{1}-1}\sum\limits_{j_2=0}^{n_2-1}...\sum\limits_{j_i=0}^{n_i-1}\sum\limits_{j_{i+1}=0}^{n_{i+1}-1}a_{j_1,j_2,...j_d}\omega^{j_1k_1}_{n_1}\omega^{j_2k_2}_{n_2}...\omega^{j_ik_i}_{n_i}\omega^{j_{i+1}k_{i+1}}_{n_{i+1}}$
\end{center}
由数学归纳法可知,d维的傅里叶变换可以由再各个维度上分别做傅里叶变换得到。
\subsection{b}
我们可以用$y_{k_1,k_2,...,k_i,k_{i+1},...,k_d}=\sum\limits_{j_1=0}^{n_1-1}\sum\limits_{j_2=0}^{n_2-1}...\sum\limits_{j_d=0}^{n_d-1}a_{j_1,j_2,...j_d}\omega^{j_1k_1}_{n_1}\omega^{j_2k_2}_{n_2}...\omega^{j_dk_d}_{n_d}$中求和的顺序来表示以维度为单位计算d维DFT的顺序,比如,在此公式中,我们认为计算d维DFT的顺序是:第1维,第2维,...,第d维。
那么对于$\forall$1-d的一个排序$(s_1,s_2,...s_d)$,显然
\begin{center}
$\sum\limits_{j_1=0}^{n_1-1}\sum\limits_{j_2=0}^{n_2-1}...\sum\limits_{j_d=0}^{n_d-1}a_{j_1,j_2,...j_d}\omega^{j_1k_1}_{n_1}\omega^{j_2k_2}_{n_2}...\omega^{j_dk_d}_{n_d}=\sum\limits_{j_{s_1}=0}^{n_{s_1}-1}\sum\limits_{j_{s_2}=0}^{n_{s_2}-1}...\sum\limits_{j_{s_d}=0}^{n_{s_d}-1}a_{j_1,j_2,...j_d}\omega^{j_{s_1}k_{s_1}}_{n_{s_1}}\omega^{j_{s_2}k_{s_2}}_{n_{s_2}}...\omega^{j_{s_d}k_{s_d}}_{n_{s_d}}$
\end{center}
因此,可以证明,顺序对求和无影响。
\subsection{c}
由a知,计算第i维独立的DFT所用的时间为$\frac{n}{n_i}\mathcal{O}(n_i\lg n_i)=\mathcal{O}(n\lg n_i)$\\
因此,计算所有维的DFT所用的时间为$\mathcal{O}(\sum\limits_{i=1}^{d}(n\lg n_i))=\mathcal{O}(n\lg n)$

\end{document}
