\documentclass[UTF8]{ctexart}
\usepackage{graphicx}
\graphicspath{{img/}} 
\title{Expriment\ 3}
\author{PB17111623}
\author{PB17111623 范睿}
\date{\today}
\usepackage[a4paper,top=5cm,bottom=2.5cm]{geometry}
\usepackage{algorithm}  
\usepackage{algorithmicx}
\usepackage{amsmath}  
\usepackage{algpseudocode}  %算法的包
\usepackage{amssymb}
\begin{document}
\maketitle
\section{偏序关系}
\subsection{题目}
对于二维平面上的任意两点$A(x_1,y_1),B(x_2,y_2)$定义偏序关系$\leq$,表示$x_1 \leq x_2$ 且$y_1\leq y_2$​。现给定平面上的若干点,求最大的点的子集S使得集合中的任意两点之间均满足偏序关系$\leq$,即对$\forall A,B \in S, A\leq B$或$B\leq A$成立。只需要输出子集S的大小$|S|$。\\

输入:\\
第一行一个整数n表示点的个数。然后是n行输入,表示n个点的坐标,其中每行的格式为:$x_i\quad y_i$点的横纵坐标均为整数,且输入数据不会出现重叠的点。\\

数据规模:\\
$0< n\leq 10000$
$0<x_i,y_i\leq 10000$\\

输出:\\
输出一个整数表示最大可能的子集$|S|$的大小。\\

Sample Input:\\ 
5\\
0 1\\
1 4\\
2 5\\
3 3\\
4 2\\

Sample Output:\\
3\\
/*
{(0,1), (1,4), (2,5)}\\
*/\\
\subsection{思路}
寻找问题子结构:第i个坐标所属的拥有最大个数的满足偏序关系的集合=第i个坐标+前i-1个坐标所属的最大集合(前提是此合并后的集合满足偏序关系)\\
因此先将所有坐标从小到大排序,比较规则是:先看x分量,x分量大的坐标大,若x相同,则看y分量,y分量大的坐标大。\\
然后设一个长度为n的数组A[1...n]。A[i]记录:在前i个坐标中,包含第i个坐标的最大集合大小$A[i].num$和该集合的最大与最小元素$A[i].S$。\\
得到递归式:\\
\begin{center}
$A[i] = \max\limits_{1\leq j\leq i}(|A[j].S+(x_i, y_i)|)$
\end{center}
其中$|A[j].S+(x_i, y_i)|$的计算如下:\\
若$(x_i,y_i)$按照偏序比较规则大于等于$A[j].S$中的最大的或小于等于最小的,那么$|A[j].S+(x_i, y_i)|=|A[j].S|+1$。\\
否则说明$(x_i,y_i)$与$A[j].S$无法形成偏序关系,那么$|A[j].S+(x_i, y_i)|=1$,即集合中只有$(x_i, y_i)$一个。\\

当找到最大值对应的j时,在更新$A[i].num$的同时,还需要更新$A[i].S$,即要将$A[i].S$换成$A[j].S$和$(x_i,y_i)$组成偏序关系后的集合。\\
当i=1时,$A[i].num=1,A[i].S.(x_{min}, y_{min})=(x_i,y_i),A[i].S.(x_{max}, y_{max})=(x_i,y_i)$\\
最终答案为所有num中最大的那个。\\

排序的原因:\\
若先排了序,那么第A[i].num至少为2(i不为1时),即$(x_i,y_i)$至少可以找到一个j的集合与它组成偏序关系。
\subsection{算法}
\floatname{algorithm}{Algorithm}
\renewcommand{\algorithmicrequire}{\textbf{输入:}}
\renewcommand{\algorithmicensure}{\textbf{输出:}}
\begin{algorithm}
	\caption{Poset}
	\begin{algorithmic}[1]
	\Require 坐标个数n,每个坐标值$(x_i,y_i)\ 1\leq i \leq n$
	\Ensure 最大偏序集合大小$|S|$
	\State{Sort $(x_i,y_i,  1\leq i \leq n)$}
	\State{let A[1...n] be a new array}
	\State{$max \gets 0$}
	\For{i = 1 to n}
		\State{$A[i].num = 1$}
		\State{$A[i].S.x_{min}=x_i$}
		\State{$A[i].S.y_{min}=y_i$}
		\State{$A[i].S.x_{max}=x_i$}
		\State{$A[i].S.y_{max}=y_i$}
		\For{j = i-1 downto 1}
			\If{$\{(x_i,y_i)\}\cup A[j].S\ is\ a\ Partiallyordered\ set$}
				\If{$A[j].num \ge A[i].num$}
					\State{$A[i].S=\{(x_i,y_i)\}\cup A[j].S$}
					\State{$A[i].num \gets A[j].num+1$}
				\EndIf
			\EndIf
		\EndFor
		\If{$A[i].num \ge sum$}
			\State{$max \gets A[i].num$}
		\EndIf
	\EndFor\\
	\Return {max}
	\end{algorithmic}
\end{algorithm}

\subsection{复杂度分析}
排序利用冒泡排序,时间为$\mathcal{O}(n^2)$,遍历i从1到n,对每个i来说又有j从i-1到1,时间为$\mathcal{O}(n^2)$,因此总时间为$\mathcal{O}(n^2)$
。
\section{归并问题}
\subsection{题目}
程序员小明需要将nn个有序数组合并到一起,由于某种工程上的原因,小明只能使用一个系统函数Merge将两个相邻的数组合并为一个数组。Merge函数合并两个长度分别为$n_1,n_2$的数组的时间为$n_1+n_2$。现给定nn个数组的长度,小明需要求出最小需要的合并时间。

输入:\\
输入共两行。第一行一个整数n表示待归并的数组个数。第二行n个整数,第i个整数表示第i个数组的长度。\\

数据规模:\\
$0< n\leq 200$ 每个数组的长度均为整数,且输入数据保证最终结果范围在int32之内。\\


输出:\\
输出共一个数字,表示最小需要的归并时间。\\

Sample Input:\\
5\\
68 34 41 55 71 \\

Sample Output:\\
613\\

\subsection{思路}
寻找子问题结构:\\
与矩阵链相乘相似,若求第i到j个元素相乘的最小合并时间,将k从i遍历到j-1,用k将i...j分割成两个子序列。那么i到j元素相乘的最小合并时间转换成求得一个在k处的分割,使得i...k和k+1...j的合并时间之和加i到j所有元素相加的总值最小。\\
那么设一个二维数组A[1...n][1...n],A[i][j]存储i...j的元素相乘的最小合并时间。
那么有递归式:\\
\begin{center}
$A[i][j]=\min\limits_{i\leq k <j}(A[i][k]+A[k+1][j]+\sum_{p=i}^jn_p)$
\end{center}
边界条件为:$A[i][i]=0,1\leq i \leq n$

\subsection{算法}

\floatname{algorithm}{Algorithm}
\renewcommand{\algorithmicrequire}{\textbf{输入:}}
\renewcommand{\algorithmicensure}{\textbf{输出:}}
\begin{algorithm}
	\caption{}
	\begin{algorithmic}[1]
	\Require 每个数组长度$n_i,1\leq i\leq n$, 数组个数n
	\Ensure mintime
	\State{let A[1...n][1...n] be a new array}
	\For{i = 1 to n}
		\State{$A[i][i] \gets 0$}
	\EndFor
	\For{l = 2 to n}
		\For{i = 1 to n-l+1}
			\State{$j \gets i+l-1$}
			\State{$A[i][j]\gets Inf$}
			\State{$s \gets \sum_{p=i}^jn_p$}
			\For{k = i to j-1}
				\State{$q \gets A[i][k]+A[k+1][j]+s$}
				\If{$q < A[i][j]$}
					\State{$A[i][j]\gets q$}
				\EndIf
			\EndFor
		\EndFor
	\EndFor\\
	\Return {A[1][n]}
	\end{algorithmic}
\end{algorithm}
\subsection{复杂度分析}
三层嵌套循环,时间复杂度为$\mathcal{O}(n^3)$

\section{多重背包}
\subsection{题目}
现有一个背包可以容纳总重量为W的物品,有n种物品可以放入背包,其中每种物品单个重量为$w_i$,价值为$v_i$,可选数量为$num_i$。输出可以放入背包的物品的最大总价值。

输入:\\
第一行两个整数n,W,分别表示物品件数和背包容量。然后n行数据描述每种物品的重量、价值和可选数量。每行的格式为 $w_i\quad  v_i\quad$。

数据规模:\\
$1\leq n\leq 200,1\leq W\leq 10000,1\leq w_i\leq 1000,\leq v_i\leq 1000,\leq num_i\leq 10000$\\
所有输入数据均为整数。

输出:\\
输出一个整数表示可以装入背包的最大价值。\\

Sample Input:\\
5 100\\
7 3 68\\
10 3 161\\
10 6 55\\
5 2 14\\
3 10 165\\

Sample Output:\\
330\\

\subsection{思路}
寻找子文体结构:\\
若想求得空间为w时,存放第i到第n个物品的最优解,可以一步步尝试:\\
第i个物品放0个,value最大值=w空间放第i+1到第n个物品;第i个物品放1个,value最大值=(w-$w_i$)空间放第i+1到第n个物品的value最大值+1*$v_i$;...第i个物品放k个(k$< num_i$),value最大值==(w-$k*w_i$)空间放第i+1到第n个物品的value最大值+k*$v_i$;...第i个物品放k个(k$\ge num_i$),value最大值==(w-$num_i*w_i$)空间放第i+1到第n个物品的value最大值+$num_i$*$v_i$\\
那么可以建立一个二维数组KS[0...W][1...n+1],KS[i][j]表示当空间是i时,放第j到第n个物品的value最大值。有递归式:\\
\begin{center}
$KS[i][j]=\max\limits_{0\le k\le \lfloor\frac{j}{w_i}\rfloor}(KS[i-\max(num_i,k)*w_i][j+1]+\max(num_i,k)*v_i)$
\end{center}
边界条件为:\\
KS[i][n+1] = 0,虚拟一个数量为0,价值为0的物品,序号为n+1,那么在KS表中,它对应的那一列均为0。

\subsection{算法}

\floatname{algorithm}{Algorithm}
\renewcommand{\algorithmicrequire}{\textbf{输入:}}
\renewcommand{\algorithmicensure}{\textbf{输出:}}
\begin{algorithm}
	\caption{}
	\begin{algorithmic}[1]
	\Require 物品个数n,背包大小W,每个物品的大小,价值,个数$w_i,v_i,num_i$
	\Ensure 最大价值valuemax
	\State{let KS[0...W][1...n+1] be a new array}
	\For{i = 0 to W}
		\State{$KS[i][n+1] \gets 0$}
	\EndFor
	\For{i = n downto 1}
		\For{j = 0 to W}
			\State{$KS[j][i] \gets 0$}
			\For{$k = 0\ to\ \lfloor\frac{j}{w_i}\rfloor$}
				\If{$k > num_i$}
					\State{$value \gets v_i*num_i + KS[j-num_i*w_i][i+1]$}
				\Else
					\State{$value \gets v_i*k + KS[j-k*w_i][i+1]$}
				\EndIf
				\If{$value > KS[j][i]$}
					\State{$KS[j][i]\gets value$}
				\EndIf
			\EndFor
		\EndFor
	\EndFor\\
	\Return{KS[W][1]}
	\end{algorithmic}
\end{algorithm}

\subsection{复杂度分析}
$\mathcal{O}(W^2*n)$

\section{正方形计数}

\subsection{题目}
现有一个$n\times m$的矩形区域,其中每个单位区域可能有损坏。要求找到地面上所有不包含损坏区域的正方形的个数。

输入:\\
第一行两个整数n,m表示矩形区域的大小。接下来共有n行输入数据,每行包含m个0或1的整数,其中0表示该地面完好,1表示该地面已损坏。\\

数据规模:\\
$0<n,m\leq 2000$

输出:\\
输出一个整数表示区域内的正方形个数。输入数据保证结果不会超出int32的范围。

Sample Input:\\
5 5\\
1 0 0 0 0 \\
1 0 0 1 1 \\
0 1 0 0 0 \\
0 1 0 1 0 \\
1 0 0 0 0 \\

Sample Output:\\
18

\subsection{思路}
寻找问题子结构:\\
从左上角开始,从所往右,从上往下遍历:若当前位置是1(未损坏),则当前位置的正方形个数是0;若为0(损坏),则当前位置正方形个数为其左边、上边、左上角的三个位置的正方形个数的最小值+1,于是定义二维数组A[0...n+1][0...m+1]有递归式:\\
\begin{center}
$A[i][j]=\min(A[i-1][j],A[i][j-1],A[i-1][j-1])+1,i,j\neq 0$
\end{center}
边界条件为:\\
在给定的n*m的矩阵外部加一圈1,构成(n+1)*(m+1)的矩阵,那么二维数组A的最外围一圈就是0\\
最终结果就是所有位置的正方形个数的和。

\subsection{算法}

\floatname{algorithm}{Algorithm}
\renewcommand{\algorithmicrequire}{\textbf{输入:}}
\renewcommand{\algorithmicensure}{\textbf{输出:}}
\begin{algorithm}
	\caption{dangerous goods}
	\begin{algorithmic}[1]
	\Require 行数n,列数m,矩形区域M[1...n][1...m]
	\Ensure 区域内正方形数sum
	\State{let A[0...n+1][0...m+1] be a new array}
	\For{i = 0 to n+1}
		\State{$A[i][0] \gets 0$}
		\State{$A[i][m+1] \gets 0$}
	\EndFor
	\For{i = 0 to m+1}
		\State{$A[0][i] \gets 0$}
		\State{$A[n+1][i] \gets 0$}
	\EndFor
	\State{$sum \gets 0$}
	\For{i = 1 to n}
		\For{j = 1 to m}
			\If{M[i][j] is 1}
				\State{$A[i][j]\gets 0$}
			\Else
				\State{$A[i][j] \gets \min(A[i-1][j],A[i][j], A[i-1][j-1])+1$}
				\State{$sum \gets sum+A[i][j]$}
			\EndIf
		\EndFor
	\EndFor\\
	\Return{sum}
	\end{algorithmic}
\end{algorithm}

\subsection{复杂度分析}
$\mathcal{O}(n\times m)$

\end{document}