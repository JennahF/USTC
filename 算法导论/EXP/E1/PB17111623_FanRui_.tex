\documentclass[UTF8]{ctexart}
\usepackage{graphicx}
\graphicspath{{img/}} 
\title{Expriment\ 1}
\author{PB17111623}
\author{PB17111623 范睿}
\date{\today}
\usepackage[a4paper,top=3cm,bottom=2.5cm]{geometry}
\usepackage{algorithm}  
\usepackage{algorithmicx}
\usepackage{amsmath}  
\usepackage{algpseudocode}  %算法的包
\usepackage{amssymb}
\begin{document}
\maketitle
\section{数据库查询v1}
\subsection{思路}
INSERT:\quad一次性读入 所有INSERT命令并解码所有被插入的数字\\
FIND:\quad 读入第一个FIND时,对所有数据进行快速排序\\
利用二分查找返回FIND的结果\\
EXIT:\quad 遇到EXIT退出
\subsection{算法}
\floatname{algorithm}{数据库查询v1}
\renewcommand{\algorithmicrequire}{\textbf{输入:}}
\renewcommand{\algorithmicensure}{\textbf{输出:}}
\begin{algorithm}
	\caption{Database}
	\begin{algorithmic}
	\Require $Instructions$全部指令
	\Ensure $OUT$查找结果集合
	\While{InsType is not $EXIT$}
		\If{InsType is $INSERT$}
		\State //Insert\ the\ id\ and\ the\ attribute\ without\ sorting
		\ElsIf{InsType is $FIND$}
			\If {firstime\_finding}
				\State Random\_Sort(database)
			\EndIf
			\If {!Binary\_Search(database, queried\_data, 0, database.length)} 
				\State PRINT(-1)
			\EndIf
		\EndIf
		
	\EndWhile
	\end{algorithmic}
\end{algorithm}
\subsection{复杂度分析}
二分查找复杂度为{$\mathcal{O}(\log n)$}\\
快速排序平均时间复杂度为{$\mathcal{O}(n\log n)$},最坏时间复杂度为{$\mathcal{O}(n^2)$}\\
此算法平均时间复杂度为{$\mathcal{O}(n\log n+\log n)=\mathcal{O}(n\log n)$}\\
最坏时间复杂度为{$\mathcal{O}(n^2+\log n)=\mathcal{O}(n^2)$}
\newpage
\section{股票}
\subsection{思路}
此题本质上是在求一个数串逆序数的大小\\
利用分治法的算法特性:\\
每次在分治法merge时,传入的p、q、r分别是左串的左端点、左串的右端点(q+1为右串的左端点),右串的右端点。
其中左串与右串都是从左到右排好序的串。\\
merge每次选择整个串中最小的元素加入最后的序列中。如果选择的是右串的元素,那么说明左串的未选择的所有元素比右串的被选择的元素都大,那么在最后结果sum中加上左串中所有的未选择的元素个数;如果选择的是左串的元素,不改变sum的值。(因为左串天然地存在于右串的前面)
\subsection{算法}
\floatname{algorithm}{股票算法}
\renewcommand{\algorithmicrequire}{\textbf{输入:}}
\renewcommand{\algorithmicensure}{\textbf{输出:}}
\begin{algorithm}
	\caption{Merge}
	\begin{algorithmic}[1]
	\Require $Array$数组,$p$左串左端点,$q$左串右端点,$r$右串右端点
	\Ensure $sum$\ p与r间的逆序数
	\State sum = 0
	\For {$i = p$ to $q$}
		\State Left[i-p] = Array[i]
	\EndFor
		\For {$i = q+1$ to $r$}
		\State Right[i-q-1] = Array[i]
	\EndFor
	\State i=0
	\State j=0
	\For {$k = p$ to $r$}
		\If{Left[i]\ $\leq$\ Right[j]}
			\State Array[k] = Left[i]
			\State i++
		\Else
			\State Array[k] =Right[j]
			\State j++
			\State sum += Left.length - i - 1
		\EndIf
	\EndFor
	\end{algorithmic}
\end{algorithm}

\subsection{复杂度分析}
此算法仅在分治法的基础上加了几条命令,所以其复杂度与分治法的复杂度相同。\\
所以此算法的复杂度为{$\mathcal{O}(n\log n)$}

\section{弹幕游戏}
\subsection{思路}
先将数字串利用快速排序排好序\\
再遍历计算相邻两个数字的差,同时找出最大的差值。
\subsection{算法}
\floatname{algorithm}{弹幕游戏算法}
\renewcommand{\algorithmicrequire}{\textbf{输入:}}
\renewcommand{\algorithmicensure}{\textbf{输出:}}
\begin{algorithm}
	\caption{Bullet Game}
	\begin{algorithmic}[1]
	\Require $n$数列长度,$Array$坐标
	\Ensure  $d$最大值
	\State Quick\_Sort(Array, 0, Array.length)
	\State max = 0
	\For {$i = 1$ to $Array.length-1$}
		\If{max $\leq$ Array[i+1]-Array[i]}
			\State max = Array[i+1] - Array[i]
		\EndIf
	\EndFor
	\State d = max
	\end{algorithmic}
\end{algorithm}
\subsection{复杂度分析}
快速排序平均时间复杂度为{$\mathcal{O}(n\log n)$},最坏时间复杂度为{$\mathcal{O}(n^2)$}\\
遍历时间复杂度为{$\mathcal{O}(n)$}\\
所以此算法平均时间复杂度{$\mathcal{O}(n\log n+n)=\mathcal{O}(n\log n)$}\\
最坏时间复杂度为{$\mathcal{O}(n^2+ n)=\mathcal{O}(n^2)$}
\newpage
\section{银行卡}
\subsection{思路}
此题本质上是为了计算出每一个大小为k的窗口中最大的元素。\\
利用双端队列解决:\\
定义一个双端队列:其队3头可以出队,队尾可以入队+出队。\\
理想情况下,在窗口在第p个位置时,从p开始往后k个元素的最大值始终处于队头。\\
①第一个任务是初始化此队列,使当窗口在第0个位置(窗口框住0...k+1的元素)时,队列具有上述性质。\\
实现方案:用i遍历数组0到k-1的元素:\\
1.若队列为空,将i入队;\\
2.若队列不为空,且若数组中第i个位置的元素比第队尾个位置的元素大,从队尾出队;\\
3.若队列不为空,且若数组中第i个位置的元素比第队尾个位置的元素小,将i入队。(表示第i个位置的元素在前面所有元素滑出窗口后有可能成为最大值)\\
②第二个任务是遍历数组,找出窗口在每个位置时的窗口内最大元素,假设窗口在第p个位置(p=0...n-k+1):\\
1.数组中第p+k-1个元素为每次窗口移动后新进来的元素,令每次新进来的元素为x,若x大于数组中队尾位置的元素,从尾部出队。\\
2.数组中第p+k-1个元素为每次窗口移动后新进来的元素,令每次新进来的元素为x,若x小于数组中队尾位置的元素,将p+k-1入队。\\
3.若队头元素小于i,说明队头元素已经滑出窗口,从队头出队。\\
4.若队头元素大于等于i,打印数组中队头位置的值。
\subsection{算法}
\floatname{algorithm}{银行卡算法}
\renewcommand{\algorithmicrequire}{\textbf{输入:}}
\renewcommand{\algorithmicensure}{\textbf{输出:}}
\begin{algorithm}
	\caption{VISA}
	\begin{algorithmic}[1]
	\Require $n$数列长度,$k$窗口大小,$Array$每天的钱数
	\Ensure  $d$最大值
	\For {$i=1$ to $k$}
		\If{Q is not empty}
			\State EnQ(Q, i)
		\Else
			\While{Q is not empty and Array[Q.back]$\leq$ Array[i]}
				\State DeQ\_rear(Q)
			\EndWhile
			\State EnQ(Q, i)
		\EndIf
	\EndFor
	\For {$i=1$ to $n-k$}
		\If {$i$ is not 1}
			\If{Q is not empty}
				\State EnQ(Q, i)
			\Else
				\While{Q is not empty and Array[Q.back]$\leq$ Array[i]}
					\State DeQ\_rear(Q)
				\EndWhile
				\State EnQ(Q, i)
			\EndIf
		\EndIf
	\While {Q.front $\textless$ i}
		\State DeQ\_front(Q)
	\EndWhile
	\State Print Array[Q.front]
	\EndFor
	\end{algorithmic}
\end{algorithm}
\subsection{复杂度分析}
第一个任务初始化队列的复杂度为{$\mathcal{O}(k)$}\\
第二个任务复杂度为{$\mathcal{O}(n)$}\\
因此此算法复杂度为{$\mathcal{O}(n+k)$}
\end{document}