\documentclass[UTF8]{ctexart}
\usepackage{graphicx}
\graphicspath{{img/}} 
\title{Expriment\ 2}
\author{PB17111623}
\author{PB17111623 范睿}
\date{\today}
\usepackage[a4paper,top=3cm,bottom=2.5cm]{geometry}
\usepackage{algorithm}  
\usepackage{algorithmicx}
\usepackage{amsmath}  
\usepackage{algpseudocode}  %算法的包
\usepackage{amssymb}
\begin{document}
\maketitle
\section{数据库查询v2}
\subsection{题目}
勤奋的小明为了预习下学期的数据库课程,决定亲自实现一个简单的数据库系统。该数据库系统需要处理用户的数据库插入和查询语句,并输出相应的输出。
具体来说,用户的输入共包含若干条插入语句和查询语句。其中每条插入语句包含一个非负整数表示需要插入的数据。每条查询语句包含一个整数表示待查询的键值,若该键值存在则直接输出该键值,否则输出数据库中比该键值小的最大键值。\\

输入:\\
首先是若干行插入语句,每行的格式为如下的一种:\\
INSERT keykey\\
FIND keykey\\
最后单独的一行EXIT表示输入结束。\\

数据规模:\\
插入语句和查询语句一共不超过2,000,000条。\\
$0\leq key \leq 10^9$\\

输出:\\
对每条查询语句输出一行,每行输出1个数字,表示查询的结果。\\
该键值存在则直接输出该键值,否则输出数据库中比该键值小的最大键值。\\

Sample Input:\\ 
INSERT 7\\
INSERT 11\\
INSERT 2\\
FIND 2\\
INSERT 5\\
INSERT 3\\
FIND 4\\
FIND 7\\
EXIT\\

Sample Output:\\
2\\
3\\
7\\
\subsection{思路}
利用红黑树的数据结构。\\
遇到INSERT: 插入红黑树,调整红黑树\\
遇到FIND: 利用红黑树的二叉搜索树的性质快速找到该键。若该键值不在树中,则先找到该插入位置的父节点,若该节点比其理论上的父节点键值大,输出父节点的键值,否则输出父节点的前驱的键值。
\subsection{算法}
\floatname{algorithm}{Algorithm}
\renewcommand{\algorithmicrequire}{\textbf{输入:}}
\renewcommand{\algorithmicensure}{\textbf{输出:}}
\begin{algorithm}
	\caption{Insert}
	\begin{algorithmic}[1]
	\Require RBTree T, key
	\State{$//node \gets malloc\ a\ memory\ for\ key$}
	\State{$RBInsert(T, node)$}
	\State{$RBAdjust(T, node)$}
	\end{algorithmic}
\end{algorithm}

\begin{algorithm}
	\caption{Find}
	\begin{algorithmic}[1]
	\Require RBTree T, key
	\If {T.data is key}
		\State{print key}
		return
	\EndIf
	\If {T.data $\leq$ key}
		\If{T.right is NIL} 
			\State{print T.data}
		\Else
			\State{Find(T.right, key)}
		\EndIf
	\Else
		\If{T.left is NIL} 
			\State{print Precessor(T).data}
		\Else
			\State{Find(T.left, key)}
		\EndIf
	\EndIf
	\end{algorithmic}
\end{algorithm}

\subsection{复杂度分析}
INSERT和红黑树的Insert时间复杂度一样,为$\mathcal{O}(\log n)$\\
FIND找到该插入的位置复杂度为$\mathcal{O}(\log n)$,找前驱复杂度为$\mathcal{O}(\log n)$,因此总复杂度为$\mathcal{O}(\log n)$

\section{军训排队}
\subsection{题目}
现有n个学生排成一个固定队伍进行军训,教官小明有一份所有n个人的名单(不同的人可能重名)。小明想要在整个队伍中找到一个连续的子队伍,并且满足该子队的所有人恰好有k个不重复的名字。请帮小明计算一下一共有多少种可能的子队伍。\\

输入:\\
一共有两行,第一行有两个数字n和k,用空格分隔第二行有n个单词$name_i$​,用空格分隔输入保证$len(name_i)\leq5$(即输入的名字最多只含有5个字符)

数据规模:

$0<k<n\ \leq\ 10,000,0000$

输出:\\
输出一共一个数字,即可能的子队伍数量。\\

Sample Input:\\
5 2\\
yyyyy iiiii yyyyy iiiii sssss\\

Sample Output:\\
7\\
/*\\
一共7种可能的子队伍为如下:\\
yyyyy iiiii\\
iiiii yyyyy\\
yyyyy iiiii\\
iiiii sssss\\
yyyyy iiiii yyyyy\\
iiiii yyyyy iiiii\\
yyyyy iiiii yyyyy iiiii\\
*/\\

\subsection{思路}
利用滑动窗口。low,mid,high标记窗口。low$\leq$mid$\leq$high。维持low和mid之间的名字种类有k个,low和high之间的名字种类有k+1个。每当找到这样的组合,可以计算出:以low开始,不同名字的个数有k个的连续组合共有high-mid+1个,然后将low+1,继续寻找。\\
查找方法:\\
由于名字的位数最多有5位,将所有情况的个数计算出来,总数=26+26*26+26*26*26+\\26*26*26*26+26*26*26*26*26$\leq$2000000,因此定义一个20000000的LUT,若改名字出现了,设置对应位置为1。

\subsection{算法}

\floatname{algorithm}{Algorithm}
\renewcommand{\algorithmicrequire}{\textbf{输入:}}
\renewcommand{\algorithmicensure}{\textbf{输出:}}
\begin{algorithm}
	\caption{}
	\begin{algorithmic}[1]
	\Require n, k, names
	\Ensure sum
	\State{mid $\gets$ 1}
	\State{NameNum $\gets$ 0}
	\State{sum $\gets$ 0}
	\For{low = 1 to n}
		\State{mid = max(mid, low)}
		\While{mid $\leq$ n and NameNum $\leq$ k}
			\State{//Add names[mid] into Queue}
			\State{mid $\gets$ mid+1}
		\EndWhile
	\If{NameNum $\leq$ k} 
		\State{return sum}
	\EndIf
	\State{high $\gets$ mid}
	\While{high $\leq$ n and NameExists(names[high])}
		\State{high $\gets$ high+1}
	\EndWhile
	\State{sum += high-mid+1}
	\State{DecreaseNumberorDelete(names[low])}
	\EndFor
	\end{algorithmic}
\end{algorithm}
\subsection{复杂度分析}
low从1走到n,为$\mathcal{O}(n)$,mid从1走到n,为$\mathcal{O}(n)$, 复杂度为$\mathcal{O}(n)$\\
\\
\\

\section{内存分配}
\subsection{题目}
C语言中需要申请一块连续的内存时需要使用malloc等函数。如果分配成功则返回指向被分配内存的指针(此存储区中的初始值不确定),否则返回空指针NULL。现在小明决定实现一个类似malloc的内存分配系统,具体来说,他需要连续处理若干申请内存的请求,这个请求用一个闭区间$[a_i..b_i]$来表示。当这个区间和当前已被申请的内存产生重叠时,则返回内存分配失败的信息。否则返回内存分配成功,并将该区间标记为已被占用。假设初始状态下内存均未被占用,管理的内存地址范围为0-1GB($0-2^{30}$)。\\

输入:\\
输入数据共n+1行。第一行一个整数n表示共需要处理n次内存分配。然后是n行数据,每行的格式为$a_i b_i$,表示申请区间为$[a_i,b_i]$\\

数据规模:

$n\leq 1,000,000$\\
$0<a_i\leq b_i \leq 2^{30}$\\

输出:\\
输出共n行。对于每行内存分配的申请,若申请成功则输出一行0,若申请失败则输出一行-1。

Sample Input:\\
5\\
0 1\\
8 9\\
5 7\\
1 2\\
2 4\\

Sample Output:\\
0\\
0\\
0\\
-1\\
0\\

\subsection{思路}
利用区间树。每个RBNode存一个区间,键值为区间的low值。每次到来一个新区间先查找有没有重叠,若没有,加入区间树,返回1,否则返回0。\\
查看重叠的方法:从根结点查找,若区间的low值大于根结点的max值,说明没哟overlap,输出1,将它加入区间树,否则观察该区间的low和根结点的键值的大小,若前者大,在根结点的右子树中再次判断,否则在根节点的左子树中再次判断。\\

\subsection{算法}

\floatname{algorithm}{Algorithm}
\renewcommand{\algorithmicrequire}{\textbf{输入:}}
\renewcommand{\algorithmicensure}{\textbf{输出:}}
\begin{algorithm}
	\caption{FindInterval}
	\begin{algorithmic}[1]
	\Require RBTree T, Interval data
	\Ensure 1 for no overlap, 0 for overlap 
	\If{T is NIL} 
		\State{return 1}
	\EndIf
	\If{data.low $\geq$ T.max}
		\State{return 1}
	\Else
		\If{Overlap(T.interval, data)}
			\State{return 0}
		\Else
			\If{data.low $\geq$ T.data.low}
				\State{return FindInterval(T.right, data)}
			\Else
				\State{return FindInterval(T.left, data)}
			\EndIf
		\EndIf
	\EndIf
	\end{algorithmic}
\end{algorithm}

\subsection{复杂度分析}
插入区间和红黑树插入+调整的复杂度相同,为$\mathcal{O}(\log n)$\\
判断重叠的复杂度为$\mathcal{O}(\log n)$

\section{危险品放置}

\subsection{题目}
现有若干危险品需要放置在A,B两个仓库。当两种特定的危险品放置在相同地点时即可能产生危险。我们用危险系数$\alpha_{i,j}$表示危险品i,j放置在一起的危险程度。一些危险品即使放置在一起也不会产生任何危险,此时$\alpha_{i,j}=0$,还有一些危险品即使单独放置也会产生危险,此时$\alpha_{i,i}>0$。定义两个仓库整体的危险系数为$\max(\max_{i,j\in A}\alpha_{i,j},\max_{i,j\in B}\alpha_{i,j})$,即放置在一起的所有危险品两两组合的危险系数的最大值。现在对于一组给定的危险系数,需要设计方案使得整体危险系数最小。

输入:\\
输入共m+1行。第一行两个整数n和m表示共有n种危险品,危险品之间的危险组合(危险系数非零的物品组合)共m种。接下来的m行,每行三个整数$i,j,\alpha_{i,j}$,表示(i,j)为危险组合(i,j可能相等),其危险系数为$\alpha_{i,j}\geq 0$。
\\
数据规模:\\

$0<n\leq 100,000$\\
$0<m\leq1,000,000$\\

输出:\\
输出共一行,包含一个整数,表示整体危险系数的最小值。\\

Sample Input:\\
3 3\\
1 2 4\\
2 3 3\\
1 3 2\\

Sample Output:\\
2\\
/*\\
将1,3放在A仓库,2放在B仓库\\
*/\\

\subsection{思路}
利用不相交集合的树形结构。
先将所有危险程度大小从大到小排序,然后从大到小一次访问。每次访问拿到i,j和$\alpha_{i,j}$,先判断i,j有无冲突。若冲突,输出$\alpha_{i,j}$,返回,否则为i,j,i',j'分别建立四个集合,然后将i和j'合并,将j和i'合并,访问下一个。\\
判断i,j是否冲突的方法是:顺着i的parent向上找到根,顺着j的parent向上找到根,若根相同,说明i和j处在同一集合,不能分开,说明冲突,否则不冲突。

\subsection{算法}

\floatname{algorithm}{Algorithm}
\renewcommand{\algorithmicrequire}{\textbf{输入:}}
\renewcommand{\algorithmicensure}{\textbf{输出:}}
\begin{algorithm}
	\caption{dangerous goods}
	\begin{algorithmic}[1]
	\Require goods' number n, sorted dangerous degree Ns[m]
	\Ensure minimun dangerous degress degree
	\State{a[n] stores $goood_i$'s pointer}
	\State{b[n] stores $goood_{i'}$'s pointer}
	\For{i = 1 to n}
		\If{a[Ns[i].i] is NULL} 
			MakeSet(a, Ns[i].i)
		\EndIf
		\If{a[Ns[i].j] is NULL} 
			MakeSet(a, Ns[i].j)
		\EndIf
		\If{b[Ns[i].i] is NULL} 
			MakeSet(b, Ns[i].i)
		\EndIf
		\If{b[Ns[i].j] is NULL} 
			MakeSet(b, Ns[i].j)
		\EndIf
		\If{FindSet(a, Ns[i].i) == FindSet(a, Ns[i].j)}
			\State{return Ns[i].alpha}
		\EndIf
		\State{Union(a[Ns[i].i], b[Ns[i].j])}
		\State{Union(a[Ns[i].j], b[Ns[i].i])}
	\EndFor
	\end{algorithmic}
\end{algorithm}

\subsection{复杂度分析}
i从1到n,为$\mathcal{O}(n)$,MakeSet时间为$\mathcal{O}(1)$,Union的操作是不相交集合的Union复杂度相同,为$\mathcal{O}(\alpha(n)), \alpha(n) < 4$,所以总复杂度为$\mathcal{O}(n)$

\end{document}